%%
% HISTORY:
%
% Daniel Sykora (26.12.2004)
%
% Pavel Tvrdik  (17.12.2004)
%
% Michal Valenta (3.8. 2005)
% upraveno pro formatovani diplomovych praci

%preamble declarations
\tolerance=500
%\raggedbottom
\textwidth=158.4mm
\textheight=241.3mm
\oddsidemargin=5mm
\evensidemargin=-5mm
\topmargin=-0.3in
\parskip=0pt
\parindent=0.5em
\setcounter{secnumdepth}{3}
\setcounter{tocdepth}{3}


\def\ifndef#1{\expandafter\ifx\csname#1\endcsname\relax} % check whether macro is defined

\newcommand\coverpagestarts
{\pagenumbering{roman}\thispagestyle{empty}\setcounter{page}{3}
\begin{center}
%\large\sffamily
\large\rmfamily
\University\\
\Faculty\\
\vglue 10mm
\includegraphics[width=50mm]{LogoCVUT}
\vglue 30mm
{\large \TypeOfWork}\\
\bigskip
{\Large\bf \WorkTitle}\\
\bigskip
\bigskip
{\Large\it \FirstandFamilyName}\\
\vfill
 \item[Vedoucí práce:]\ \Supervisor\\
\vglue 15mm
{Studijní program: \StudProgram}\\
\bigskip
{Obor: \StudBranch}\\
\bigskip
\Month\ \Year
\end{center}
\newpage
}

\newcommand\mainbodystarts{
\pagestyle{headings}\pagenumbering{arabic}\setcounter{page}{1}
}

%include any longer words that by default hyphenate improperly
% \hyphenation{di-men-sion-al ma-te-ma-ti-ky hard-ware trans-fer-able}
%\hyphenation{po-po-ka-te-pe-tl nej-dů-le-ži-těj-ší}
%\hyphenation{nej-ne-ob-hos-po-da-řo-vá-va-tel-něj-ší-mi}
%\hyphenation{ma-te-ri-ál}
%Chapters start at odd pages if you include:
%\newcommand\chapter{\if@openright\cleardoublepage\else\clearpage\fi
% into /usr/lib/texmf/tex/latex/latex/book.cls

%your specific macros

%structuring of mathematical texts
\newtheorem{lemma}{Lemma}[section]
\newtheorem{note}{Note}[section]
\newtheorem{definition}{Definition}[section]
\newtheorem{example}{Example}[section]
\newtheorem{corollary}{Corollary}[section]
\newtheorem{theorem}{Theorem}[section]
\newtheorem{proposition}{Proposition}[section]
%
\newcommand{\bcen}{\begin{center}}
\newcommand{\ecen}{\end{center}}
\newcommand{\blem}{\begin{lemma}\sl}
\newcommand{\elem}{\end{lemma}\rm}
\newcommand{\bnote}{\begin{note}\rm}
\newcommand{\enote}{\end{note}}
\newcommand{\bcor}{\begin{corollary}\sl}
\newcommand{\ecor}{\end{corollary}\rm}
\newcommand{\bdefi}{\begin{definition}\rm}
\newcommand{\edefi}{\end{definition}}
\newcommand{\btheo}{\begin{theorem}\sl}
\newcommand{\etheo}{\end{theorem}\rm}
\newcommand{\bprop}{\begin{proposition}\sl}
\newcommand{\eprop}{\end{proposition}\rm}
\newcommand{\bexam}{\begin{example}\rm}
\newcommand{\eexam}{\end{example}}
%
\newcommand{\bfig}{\begin{figure}\begin{center}}
\newcommand{\efig}{\end{center}\end{figure}}
\newcommand{\btab}{\begin{table}\begin{center}}
\newcommand{\etab}{\end{center}\end{table}}
\newcommand{\benum}{\begin{enumerate}}
\newcommand{\eenum}{\end{enumerate}}
\newcommand{\bitem}{\begin{itemize}}
\newcommand{\eitem}{\end{itemize}}
\newcommand{\bflushr}{\begin{flushright}}
\newcommand{\eflushr}{\end{flushright}}
%
\newcommand{\esceario}{\end{xtabular}\end{center}}


